\chapter{CRONOGRAMA}
\label{cap:cronograma}

O cronograma do projeto está organizado em três fases principais, conforme descrito 
no Capítulo~\ref{cap:introducao}, abrangendo o período de março de 2025 a novembro 
de 2026 (20 meses). A Tabela~\ref{tab:cronograma} apresenta as atividades planejadas 
e sua distribuição temporal.

\begin{table}[htbp]
\centering
\caption{Cronograma de atividades do projeto (março/2025 -- novembro/2026).}
\label{tab:cronograma}
\small
\begin{tabular}{clcc}
\hline
\textbf{Período} & \textbf{Atividade} & \textbf{Fase} & \textbf{Status} \\
\hline
Março/2025 & Definição sobre o tema com orientador/coorientador & Fase 1 & Concluído \\
Abril/2025 & Revisão bibliográfica inicial & Fase 1 & Concluído \\
Junho/2025 & Refatoração da revisão sistemática & Fase 1 & Concluído \\
Novembro/2025 & Leitura dos artigos + criação do PTC & Fase 1 & Em andamento \\
\hline
Fevereiro/2026 & Levantamento de requisitos da solução & Fase 2 & Planejado \\
Março/2026 & Desenvolvimento do protótipo & Fase 2 & Planejado \\
\hline
Julho/2026 & Execução dos experimentos & Fase 3 & Planejado \\
Setembro/2026 & Análise dos resultados & Fase 3 & Planejado \\
Outubro/2026 & Correção do texto do TCC & Fase 3 & Planejado \\
Novembro/2026 & Submissão do TCC & Fase 3 & Planejado \\
\hline
\end{tabular}
\end{table}

\textbf{Observações}:

\begin{itemize}
    \item \textbf{Fase 1 (PTC)}: Revisão sistemática da literatura (março--novembro/2025).
    \item \textbf{Fase 2 (TCC)}: Desenvolvimento de protótipo de ferramenta computacional 
    para educação matemática (fevereiro--junho/2026).
    \item \textbf{Fase 3 (TCC)}: Validação experimental do protótipo em contexto 
    escolar real (julho--novembro/2026).
\end{itemize}

O projeto encontra-se atualmente na reta final da Fase 1, com a documentação da 
revisão sistemática realizada conforme protocolo PRISMA 2020 \cite{PRISMA2020}. 
As fases subsequentes serão executadas durante o TCC propriamente dito, utilizando 
os resultados e lacunas identificadas nesta primeira fase como direcionamento para 
o desenvolvimento e validação da ferramenta educacional.