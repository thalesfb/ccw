\chapter{METODOLOGIA}
\label{cap:metodologia}

\section{Protocolo da Revisão Sistemática}

Este trabalho adota a metodologia de \textbf{Revisão Sistemática da Literatura} seguindo as diretrizes PRISMA 2020 (\textit{Preferred Reporting Items for Systematic Reviews and Meta-Analyses}) \cite{PRISMA2020}. A revisão sistemática é um método estruturado e transparente para identificar, selecionar, avaliar e sintetizar estudos relevantes publicados sobre um tópico específico, minimizando vieses e garantindo reprodutibilidade.

A escolha da abordagem PRISMA justifica-se por sua ampla aceitação na comunidade científica, seu rigor metodológico e sua capacidade de assegurar que o processo de revisão seja explícito, replicável e auditável. Esta metodologia é especialmente adequada para a Fase 1 do projeto, na qual o objetivo principal é mapear o estado da arte das técnicas computacionais aplicadas à educação matemática.

\section{Estratégia de Busca}

\subsection{Bases de Dados e APIs}

A coleta de dados foi realizada mediante integração automatizada com as APIs de quatro bases científicas complementares:

\begin{enumerate}
    \item \textbf{Semantic Scholar}: Ampla cobertura em ciência da computação com métricas de influência científica;
    \item \textbf{OpenAlex}: Base de dados aberta e abrangente, sucessora do Microsoft Academic Graph;
    \item \textbf{Crossref}: Foco em metadados precisos de publicações e identificadores DOI;
    \item \textbf{CORE}: Agregador especializado em artigos de acesso aberto.
\end{enumerate}

A integração de múltiplas fontes proporciona: (i)~cobertura complementar, pois cada base possui especialidades distintas; (ii)~redução de viés de seleção; (iii)~reprodutibilidade mediante automação; e (iv)~eficiência na coleta sistemática de grandes volumes de dados.

\subsection{Estratégia de Busca Bilíngue}

A estratégia de busca combina três camadas de termos (base matemática, técnicas computacionais e domínio educacional), utilizando o operador booleano \texttt{AND} para garantir precisão e relevância temática.

\textbf{Camada 1 -- Base Matemática}:
\begin{itemize}
    \item Inglês: \textit{mathematics}, \textit{math} (2 termos)
    \item Português: \textit{matemática} (1 termo)
\end{itemize}

\textbf{Camada 2 -- Técnicas Computacionais} (12 termos para ambos os idiomas, com equivalentes em português):
\begin{itemize}
    \item \textit{adaptive} (adaptivo), \textit{personalized} (personalizado), \textit{tutoring} (tutor), \textit{analytics} (analítica), \textit{mining} (mineração), \textit{machine learning} (aprendizado de máquina), \textit{ai} (ia), \textit{assessment} (avaliação), \textit{student modeling} (modelagem do aluno), \textit{predictive} (preditivo), \textit{intelligent tutor} (tutor inteligente), \textit{artificial intelligence} (inteligência artificial)
\end{itemize}

\textbf{Camada 3 -- Domínio Educacional}:
\begin{itemize}
    \item Inglês: \textit{education}, \textit{learning} (2 termos)
    \item Português: \textit{educacao}, \textit{ensino} (2 termos)
\end{itemize}

A estratégia resultou em \textbf{72 consultas únicas}:
\begin{itemize}
    \item 48 consultas em inglês (2 termos base $\times$ 12 técnicas $\times$ 2 educacionais)
    \item 24 consultas em português (1 termo base $\times$ 12 técnicas $\times$ 2 educacionais)
\end{itemize}

Cada consulta segue o formato: \texttt{"termo\_base" AND "termo\_tecnica" AND "termo\_educacional"}. 

\textbf{Exemplos}:
\begin{itemize}
    \item \texttt{"mathematics" AND "machine learning" AND "education"}
    \item \texttt{"math" AND "intelligent tutor" AND "learning"}
    \item \texttt{"matemática" AND "aprendizado de máquina" AND "educa\c{c}ao"}
\end{itemize}

Esta abordagem de \textbf{expansão em 3 camadas} garante:
\begin{enumerate}
    \item \textbf{Precisão temática}: Todas as queries combinam matemática + técnica computacional + domínio educacional
    \item \textbf{Cobertura abrangente}: 12 termos técnicos capturam diferentes áreas de IA/ML/LA
    \item \textbf{Inclusão bilinguística}: Queries em inglês e português ampliam representação geográfica
    \item \textbf{Reprodutibilidade}: Estrutura documentada em \texttt{search\_terms.py} (módulo canônico)
\end{enumerate}

\section{Critérios de Seleção (PICOS)}

Os critérios de inclusão e exclusão foram definidos conforme o framework PICOS:

\begin{itemize}
    \item \textbf{P (Population)}: Estudantes de matemática em qualquer nível educacional;
    \item \textbf{I (Intervention)}: Aplicação de técnicas computacionais (ML, IA, LA, STI, NLP, etc.);
    \item \textbf{C (Comparison)}: Abordagens pedagógicas tradicionais ou alternativas (quando aplicável);
    \item \textbf{O (Outcomes)}: Desempenho acadêmico, diagnóstico de competências, personalização do ensino;
    \item \textbf{S (Study Design)}: Estudos empíricos, quasi-experimentais ou estudos de caso com evidências práticas.
\end{itemize}

\subsection{Critérios de Inclusão}

\begin{enumerate}
    \item Artigos completos revisados por pares (\textit{peer-reviewed});
    \item Publicações entre 2015 e 2025 (últimos 10 anos);
    \item Foco explícito em técnicas computacionais aplicadas à educação matemática;
    \item Apresentação de dados empíricos, metodologias detalhadas ou evidências de desenvolvimento/avaliação de sistemas;
    \item Idiomas: inglês ou português.
\end{enumerate}

\subsection{Critérios de Exclusão}

\begin{enumerate}
    \item Estudos com metodologia insuficiente ou incoerente;
    \item Trabalhos com foco indireto ou descontextualizado da matemática;
    \item Publicações predominantemente teóricas sem suporte empírico;
    \item Estudos com impacto não mensurável ou irrelevante;
    \item Documentos não validados cientificamente (\textit{preprints}, relatórios internos);
    \item Publicações com falhas conceituais ou contradições metodológicas;
    \item Idiomas diferentes de inglês ou português.
\end{enumerate}

\section{Processo de Seleção (PRISMA)}

O fluxo de seleção dos estudos seguiu rigorosamente as etapas PRISMA:

\subsection{Identificação}

Execução automatizada das 72 consultas nas quatro APIs, resultando na coleta inicial de \textbf{9.513 registros}. Cada registro inclui metadados bibliográficos (título, autores, ano, \textit{venue}, DOI/URL) e, quando disponível, resumo (\textit{abstract}).

\subsection{Triagem}

Aplicação de deduplicação automática e filtros para remover:
\begin{itemize}
    \item Registros duplicados identificados via cache;
    \item Sem título ou resumo válido;
    \item Fora do período 2017--2026;
    \item Em idiomas não compatíveis.
\end{itemize}

Nesta etapa foram excluídos \textbf{2.565 registros duplicados} (27,0\%), resultando em \textbf{6.948 estudos únicos}. Após aplicação de critérios de triagem, \textbf{1.945 estudos} avançaram para análise de elegibilidade.

\subsection{Elegibilidade}

Avaliação dos estudos mediante sistema de pontuação (\textit{scoring}) multi-critério baseado em:

\begin{enumerate}
    \item \textbf{Técnicas Computacionais} (0--3 pontos): Presença e relevância de termos relacionados a ML, IA, LA, STI, etc.;
    \item \textbf{Contexto Educacional Matemático} (0--3 pontos): Aderência explícita ao domínio da educação matemática;
    \item \textbf{Qualidade de Metadados/Abstract} (0--2 pontos): Completude e clareza das informações bibliográficas;
    \item \textbf{Impacto e Acesso} (0--2 pontos): Disponibilidade de DOI, acesso aberto, número de citações.
\end{enumerate}

A pontuação total varia de 0 a 10. O \textit{threshold} de inclusão foi definido em $\text{relevance\_score} \geq 4.0$, garantindo seleção rigorosa de estudos com aderência temática e metodológica adequadas.

Nesta etapa foram excluídos \textbf{1.929 registros} (99,2\% dos elegíveis), refletindo o rigor dos critérios de seleção.

\subsection{Inclusão}

Após aplicação do \textit{threshold}, foram incluídos \textbf{16 estudos} para síntese qualitativa, representando uma taxa de inclusão de aproximadamente \textbf{0,17\%} em relação ao total identificado (16/9.513).

\section{Deduplicação}

A deduplicação foi realizada em dois níveis:

\begin{enumerate}
    \item \textbf{Por DOI}: Registros com DOI idêntico foram unificados, priorizando a fonte com maior completude de metadados;
    \item \textbf{Por Similaridade de Título}: Títulos com similaridade TF-IDF coseno $>0.9$ foram considerados duplicatas, mantendo-se apenas o registro mais completo.
\end{enumerate}

Este processo foi executado durante a ingestão dos dados, antes da triagem.

\section{Infraestrutura Tecnológica}

\subsection{Pipeline Automatizado}

O pipeline de revisão sistemática foi implementado em \textbf{Python 3.11+}, utilizando:

\begin{itemize}
    \item \textbf{SQLite}: Banco de dados local para armazenamento estruturado (\texttt{systematic\_review.sqlite});
    \item \textbf{Requests}: Cliente HTTP para comunicação com APIs;
    \item \textbf{Pandas}: Manipulação e análise de dados tabulares;
    \item \textbf{Scikit-learn}: Cálculo de similaridades TF-IDF para deduplicação.
\end{itemize}

O sistema implementa cache local com taxa de reutilização de aproximadamente \textbf{63\%}, reduzindo significativamente o tempo de reprocessamento e o número de requisições às APIs.

\subsection{Reprodutibilidade}

O pipeline é totalmente reprodutível mediante comandos CLI (\textit{Command Line Interface}):

\begin{verbatim}
# Executar pipeline completo
python -m research.src.cli run-pipeline --min-score 4.0

# Gerar estatísticas
python -m research.src.cli stats

# Exportar resultados
python -m research.src.cli export
\end{verbatim}

Todos os parâmetros de configuração (thresholds, APIs, termos de busca) estão documentados e versionados no repositório Git do projeto.

\section{Limitações Metodológicas}

As principais limitações identificadas incluem:

\begin{enumerate}
    \item \textbf{Viés Linguístico}: Cobertura limitada a inglês e português, excluindo potencialmente estudos relevantes em outros idiomas;
    \item \textbf{Viés Temporal}: Foco nos últimos 10 anos pode excluir estudos seminais anteriores a 2015;
    \item \textbf{Alta Taxa de Exclusão}: A taxa de inclusão de 0,17\% reflete rigor elevado, mas pode limitar a abrangência da síntese;
    \item \textbf{Dependência de APIs}: Disponibilidade e completude dos dados dependem das APIs consultadas;
    \item \textbf{Ausência de Grey Literature}: Não foram incluídas teses, dissertações ou relatórios técnicos não indexados.
\end{enumerate}

Apesar dessas limitações, a metodologia adotada garante alto grau de rigor, transparência e reprodutibilidade, alinhada às melhores práticas de revisões sistemáticas.

% 
%--------- FIM METODOLOGIA ------------
%