\chapter{INTRODUÇÃO}
\label{cap:introducao}

A educação matemática enfrenta o desafio constante de atender às diversas necessidades de aprendizagem dos alunos em salas de aula heterogêneas. A personalização do ensino, embora reconhecida como uma abordagem eficaz para melhorar o desempenho e o engajamento dos estudantes, é uma tarefa complexa e demorada para os professores \cite{Innovative2023_005, Design2025_004}. A dificuldade em diagnosticar com precisão e em tempo hábil as competências e dificuldades individuais de cada aluno representa uma barreira significativa para a otimização dos planos de ensino \cite{Authentic2024_013}.

Nesse contexto, as tecnologias computacionais — como \textit{machine learning}, análise de dados educacionais (\textit{learning analytics}) e inteligência artificial — surgem como aliadas poderosas \cite{Machine2019_007, Implementation2025_000}. Essas técnicas oferecem o potencial de automatizar a avaliação diagnóstica e fornecer aos educadores informações estruturadas e acionáveis para intervenções pedagógicas mais direcionadas e eficazes \cite{Math2021_001, Multimodels2020_002}. Estudos recentes demonstram que sistemas adaptativos baseados em dados podem melhorar significativamente o desempenho estudantil e reduzir o tempo necessário para alcançar objetivos de aprendizagem \cite{Assessing2024_015}.

\section{Justificativa}

A relevância deste projeto reside no seu potencial de impacto tanto para educadores quanto para alunos. Para os professores, a síntese sistemática das técnicas computacionais aplicadas à educação matemática oferece fundamentação científica sólida para a tomada de decisão sobre investimentos em tecnologia educacional e adoção de práticas baseadas em evidências \cite{Identifying2017_006}. Além disso, a identificação de abordagens eficazes pode orientar o desenvolvimento de ferramentas que reduzam a carga de trabalho manual de avaliação e forneçam \textit{insights} acionáveis baseados em dados, permitindo que os educadores dediquem mais tempo à elaboração de estratégias pedagógicas personalizadas \cite{Performance2023_014}.

Para os alunos, um ensino adaptado às suas necessidades individuais — fundamentado em diagnósticos automatizados precisos — pode resultar em melhoria significativa na aprendizagem, na motivação e na redução da ansiedade em relação à matemática \cite{Assessing2024_015}. Estudos reportam ganhos de aprendizagem de 10 a 20\% em sistemas adaptativos quando comparados a abordagens tradicionais, além de redução de até 20\% no tempo necessário para atingir objetivos pedagógicos \cite{Design2025_004}.

Do ponto de vista acadêmico, a literatura sobre IA educacional encontra-se fragmentada em diversas áreas, carecendo de consenso terminológico. A aplicação da metodologia PRISMA 2020 garante rigor, transparência e reprodutibilidade, preenchendo lacunas deixadas por revisões anteriores que não utilizaram pipelines automatizados.

Praticamente, este trabalho orienta desenvolvedores de software e formuladores de políticas públicas com evidências empíricas, além de fundamentar cientificamente o desenvolvimento do protótipo nas fases subsequentes do TCC, garantindo escolhas técnicas e pedagógicas respaldadas por dados \cite{Data2020_011, Machine2024_009}.

O acompanhamento individualizado do progresso dos alunos em matemática é fundamental para um ensino de qualidade. No entanto, professores frequentemente carecem de ferramentas eficientes para diagnosticar as competências específicas de cada estudante em larga escala \cite{Analysis2022_003}, o que dificulta a adaptação dos planos de ensino às necessidades reais da turma. A ausência de um sistema automatizado para essa finalidade leva a um ensino mais generalizado, que pode não atender adequadamente nem os alunos com dificuldades nem os mais avançados.

Apesar do crescente número de pesquisas sobre inteligência artificial aplicada à educação matemática, a literatura científica apresenta-se fragmentada e dispersa em múltiplas bases de dados, subdisciplinas e abordagens metodológicas. Pesquisadores e desenvolvedores enfrentam dificuldades para identificar quais técnicas computacionais são mais adequadas para diferentes objetivos pedagógicos e quais abordagens foram efetivamente validadas em contextos educacionais reais \cite{Computational2017_008}.

\section{Problema de Pesquisa}

Diante desse contexto, o presente trabalho aborda o seguinte problema de pesquisa:

\textbf{Como identificar e sintetizar, através de revisão sistemática da literatura, as principais técnicas computacionais aplicadas à educação matemática, de modo a fundamentar o desenvolvimento de uma ferramenta que auxilie professores na otimização de seus planos de ensino a partir da identificação automatizada das competências e dificuldades individuais dos alunos?}

\section{Objetivos}

\subsection{Objetivo Geral}
	
Mapear e analisar sistematicamente as aplicações de técnicas computacionais — especialmente \textit{machine learning}, \textit{learning analytics} e sistemas de tutoria inteligente — no contexto da educação matemática, identificando tendências, lacunas de pesquisa e oportunidades para o desenvolvimento de um modelo computacional (MVP) que auxilie professores na personalização do ensino e no diagnóstico de competências.

\subsection{Objetivos Específicos} 

\begin{itemize}
    \item \textbf{OE1}: Realizar revisão sistemática da literatura seguindo o protocolo \cite{PRISMA2020} para identificar estudos que apliquem técnicas computacionais na educação matemática, publicados nos últimos 10 anos (2015-2025).
    
    \item \textbf{OE2}: Identificar e categorizar as principais abordagens de inteligência artificial (Machine Learning, Deep Learning, NLP, Educational Data Mining) aplicadas à educação matemática reportadas na literatura científica.
    
    \item \textbf{OE3}: Classificar as aplicações identificadas segundo suas finalidades pedagógicas: tutoria inteligente, diagnóstico de dificuldades, avaliação automatizada, personalização de conteúdo, predição de desempenho e feedback adaptativo.
    
    \item \textbf{OE4}: Analisar criticamente as metodologias de avaliação utilizadas para validar a eficácia de sistemas computacionais em contextos educacionais, identificando boas práticas e limitações metodológicas.
    
    \item \textbf{OE5}: Mapear sistematicamente as lacunas de pesquisa, limitações técnicas e desafios reportados nos estudos incluídos, propondo direções prioritárias para o desenvolvimento de ferramentas computacionais eficazes em educação matemática.
    
    \item \textbf{OE6}: Criar um pipeline automatizado e reproduzível para coleta, processamento e análise de literatura científica, contribuindo para futuras revisões sistemáticas no campo e fundamentando o desenvolvimento do protótipo nas fases subsequentes do TCC.
\end{itemize}



\section{Estrutura do Trabalho}

Este Projeto de Trabalho de Conclusão de Curso (PTC) representa a \textbf{Fase 1} de um projeto de pesquisa dividido em três etapas sequenciais:

\subsection{Fase 1: Revisão Sistemática da Literatura (PTC)}

A presente fase, correspondente ao PTC, realiza uma revisão sistemática seguindo o protocolo PRISMA 2020 para mapear o estado da arte das técnicas computacionais aplicadas à educação matemática. Esta fase é fundamentalmente \textbf{exploratória e analítica}, tendo como objetivo:

\begin{itemize}
    \item Identificar quais técnicas de IA, machine learning e learning analytics têm sido aplicadas na educação matemática
    \item Analisar os resultados reportados e metodologias de avaliação utilizadas
    \item Mapear lacunas de pesquisa e oportunidades de desenvolvimento
    \item Criar uma base de conhecimento estruturada para orientar as fases seguintes
\end{itemize}

\textbf{Entrega}: Relatório de revisão sistemática contendo: protocolo de pesquisa, análise de 17 estudos incluídos (de 9.431 identificados, 6.937 únicos após deduplicação), síntese narrativa das aplicações, mapeamento de lacunas e diretrizes para desenvolvimento de ferramentas educacionais.

\subsection{Fase 2: Desenvolvimento do Protótipo (TCC)}

A Fase 2, a ser desenvolvida no TCC, consistirá no \textbf{projeto e implementação} de um protótipo funcional de ferramenta computacional para diagnóstico de competências matemáticas, fundamentado nos achados da revisão sistemática. Atividades previstas:

\begin{itemize}
    \item Levantamento de requisitos funcionais e não-funcionais baseado na literatura
    \item Definição de arquitetura de software e escolha de tecnologias
    \item Implementação de algoritmos de machine learning para diagnóstico automatizado
    \item Desenvolvimento de interface para professores e alunos
    \item Integração com bases de dados educacionais e sistemas de gestão escolar
\end{itemize}

\subsection{Fase 3: Validação Experimental (TCC)}

A Fase 3, também parte do TCC, consistirá na \textbf{validação empírica} do protótipo desenvolvido em contexto educacional real. Atividades previstas:

\begin{itemize}
    \item Planejamento de estudo experimental (design, amostra, instrumentos)
    \item Coleta de dados em ambiente escolar controlado
    \item Análise quantitativa e qualitativa dos resultados
    \item Avaliação de eficácia, usabilidade e aceitação pelos usuários
    \item Refinamento do protótipo com base nos resultados
\end{itemize}

\subsection{Organização do Documento}

O presente documento (PTC - Fase 1) está organizado da seguinte forma:

\begin{description}
    \item[Capítulo 1 - Introdução:] Apresenta a contextualização do tema, o problema de pesquisa, os objetivos, a justificativa e a estrutura do trabalho (este capítulo).
    
    \item[Capítulo 2 - Fundamentação Teórica:] Apresenta os conceitos fundamentais sobre Inteligência Artificial, Machine Learning, Learning Analytics, Educação Matemática e Revisões Sistemáticas que embasam teoricamente a pesquisa.
    
    \item[Capítulo 3 - Metodologia:] Descreve em detalhes a metodologia PRISMA 2020 aplicada, incluindo estratégias de busca bilíngue (72 consultas em 4 APIs), critérios de seleção PICOS, processo de triagem e extração de dados.
    
    \item[Capítulo 4 - Resultados Esperados:] Apresenta os resultados preliminares da revisão sistemática, incluindo fluxo PRISMA (9.431 identificados $\rightarrow$ 6.937 únicos $\rightarrow$ 17 incluídos), síntese dos estudos incluídos (Tabela 1), análise temática das aplicações de IA identificadas, e mapeamento de lacunas de pesquisa.
    
    \item[Capítulo 5 - Cronograma:] Apresenta o planejamento temporal detalhado das três fases do projeto (Fase 1: março-novembro 2025; Fase 2: fevereiro-julho 2026; Fase 3: julho-novembro 2026).
    
    \item[Referências Bibliográficas:] Lista completa das obras citadas ao longo do trabalho, formatadas segundo normas ABNT NBR 6023:2018, incluindo os 17 estudos incluídos na revisão sistemática.
\end{description}

\subsection{Delimitação de Escopo}

É importante delimitar o escopo deste PTC:

\begin{itemize}
    \item[\checkmark] \textbf{Incluído}: Revisão sistemática, síntese narrativa, análise crítica, mapeamento de lacunas, diretrizes para desenvolvimento
    \item[\texttimes] \textbf{Não incluído}: Desenvolvimento de software, implementação de algoritmos, coleta de dados experimentais, validação empírica
\end{itemize}

O desenvolvimento e a validação da ferramenta computacional serão realizados nas Fases 2 e 3 do TCC, fundamentados nos resultados desta revisão sistemática.

% 
%--------- FIM INTRODUÇÃO------------
%