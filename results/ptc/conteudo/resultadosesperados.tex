\chapter{RESULTADOS ESPERADOS}
\label{cap:resultados}

Este capítulo apresenta os resultados preliminares obtidos mediante a aplicação 
rigorosa da metodologia PRISMA 2020, conforme descrito no Capítulo~\ref{cap:metodologia}. 
A revisão sistemática identificou 17 estudos de alta relevância (pontuação de relevância $\geq 4{,}0$) 
que constituem a base empírica para as fases subsequentes deste projeto de TCC.

\section{Fluxo PRISMA 2020}

A Figura~\ref{fig:prisma-flow} ilustra o fluxo de seleção PRISMA. O processo iniciou com a identificação de 9.431 registros. Após a remoção de duplicatas e triagem inicial, 1.883 estudos foram avaliados quanto à elegibilidade. Destes, 1.866 foram excluídos por apresentarem pontuação de relevância inferior a 4,0, resultando na inclusão final de 17 estudos (taxa de inclusão de $\sim$0,18\%). A Figura~\ref{fig:selection-funnel} apresenta essa mesma progressão em formato de funil, evidenciando a redução progressiva do conjunto de estudos em cada etapa do processo de seleção.

\begin{figure}[H]
\centering
\includegraphics[width=0.85\textwidth]{images/prisma_flow.png}
\caption{Fluxo PRISMA 2020 da revisão sistemática.}
\label{fig:prisma-flow}
\end{figure}

\begin{figure}[H]
\centering
\includegraphics[width=0.85\textwidth]{images/selection_funnel.png}
\caption{Funil de seleção de estudos.}
\label{fig:selection-funnel}
\end{figure}

\section{Estatísticas Descritivas}

A Tabela~\ref{tab:stats-descritivas} sintetiza as principais métricas quantitativas 
da revisão sistemática. Destacam-se três aspectos relevantes: (i) a elevada taxa de duplicatas (26,4\%), que evidencia a sobreposição significativa entre as bases de dados consultadas e reforça a importância da etapa de deduplicação; (ii) a taxa de exclusão na fase de elegibilidade (99,1\%), indicando que a grande maioria dos estudos inicialmente identificados não atendeu aos critérios de inclusão estabelecidos, o que é esperado em revisões sistemáticas com escopo bem delimitado; e (iii) a taxa de inclusão final de aproximadamente 0,18\%, valor consistente com revisões sistemáticas rigorosas na área de tecnologia educacional. O \textit{cache hit rate} de 92\% demonstra a eficiência do sistema de cache implementado, reduzindo significativamente o número de requisições às APIs externas.

\begin{table}[H]
\centering
\caption{Estatísticas descritivas da revisão sistemática.}
\label{tab:stats-descritivas}
\begin{tabular}{lc}
\hline
\textbf{Métrica} & \textbf{Valor} \\
\hline
Total identificado (com duplicatas) & 9.431 \\
Duplicatas removidas & 2.494 (26,4\%) \\
Registros únicos & 6.937 \\
Total elegíveis (após triagem) & 1.883 \\
Taxa de exclusão (triagem) & 72,8\% \\
Avaliados para elegibilidade & 1.883 \\
Taxa de exclusão (elegibilidade) & 99,1\% \\
Total incluído (pontuação $\geq$ 4,0) & 17 \\
Taxa de inclusão final & $\sim$0,18\% \\
Bases de dados consultadas & 4 \\
Consultas bilíngues executadas & 72 (48 EN + 24 PT) \\
Período de cobertura & 2015--2025 (11 anos) \\
Pontuação de relevância média (incluídos) & 4,2 \\
Cache hit rate & $\sim$92\% (265/287 requisições) \\
\hline
\end{tabular}
\end{table}

A Figura~\ref{fig:database-coverage} ilustra a distribuição dos registros identificados por base de dados. Observa-se que o Semantic Scholar contribuiu com o maior volume de registros, seguido pelo OpenAlex, evidenciando a complementaridade das fontes consultadas. A sobreposição entre bases justifica a elevada taxa de duplicatas (26,4\%) e reforça a importância da estratégia multi-base para garantir cobertura abrangente da literatura.

\begin{figure}[H]
\centering
\includegraphics[width=0.85\textwidth]{images/database_coverage.png}
\caption{Registros identificados por base de dados (inclui duplicatas) (n = 9.431).}
\label{fig:database-coverage}
\end{figure}

\section{Síntese dos Estudos Incluídos}

A Tabela~\ref{tab:sintese-17-estudos} apresenta a síntese dos 17 estudos que 
atenderam ao critério de relevância (pontuação $\geq 4{,}0$), organizados cronologicamente 
do mais recente ao mais antigo. A análise revela predominância de estudos focados em predição de desempenho mediante algoritmos de Machine Learning supervisionado (Random Forest, SVM, Redes Neurais), com acurácias reportadas entre 75\% e 96,89\%. Destaca-se também a presença de abordagens inovadoras como Deep Learning com Knowledge Graphs e sistemas adaptativos que reportam ganhos de aprendizagem de até 15\% e redução de 20\% no tempo de estudo.

\begin{landscape}
\scriptsize
\setlength{\tabcolsep}{3pt}\renewcommand{\arraystretch}{1.03}
\begin{longtable}{p{2.1cm}p{3.2cm}p{2.2cm}p{2.2cm}p{1.7cm}p{7.5cm}}
\caption{Síntese dos 17 estudos incluídos na revisão sistemática.}\label{tab:sintese-17-estudos}\\
\hline
\textbf{Autores/Ano} & \textbf{Título} & \textbf{Abordagem IA} & \textbf{Finalidade} & \textbf{Avaliação} & \textbf{Principais Resultados} \\
\hline
\endfirsthead
\hline
\textbf{Autores/Ano} & \textbf{Título} & \textbf{Abordagem IA} & \textbf{Finalidade} & \textbf{Avaliação} & \textbf{Principais Resultados} \\
\hline
\endhead
\hline
\multicolumn{6}{r}{\textit{Continua na próxima página}} \\
\hline
\endfoot
\hline
\endlastfoot
Tjahyadi (2025) & EDM para prever desempenho em Matemática (EF) & ML; Learning Analytics & Predição de desempenho & Performance; Statistical & SMOTERUSBoosted Trees com 75\% de acurácia \\
Zhang et al. (2025) & Caminhos personalizados com DL & DL; RL; Knowledge Graph; Sentiment & Personalização de aprendizagem & User feedback & +15\% efeito de aprendizagem; $-$20\% tempo; satisfação 4,2/5 \\
Nyantah et al. (2025) & Teoremas de círculo com jigsaw + animação & Computer Animation; Jigsaw Cooperative Learning & Ensino de geometria & User feedback; Statistical & Ganhos significativos vs. ensino tradicional ($p<0.05$) \\
Milićević et al. (2024) & Métodos de ML como apoio ao ensino & ML; Learning Analytics; Predictive Analytics & Predição / apoio ao ensino & Performance; Statistical & Heurísticas de classificação para sucesso matemático \\
Zhang (2023) & Ensino inteligente em Matemática Superior & ACO+CNN; semi-superv. CRF & Personalização/Ensino & User feedback; Statistical & Pós-teste +9,317 ($p<0.05$); ganhos vs. controle \\
Jose et al. (2024) & Sistemas adaptativos K-12 & Adaptive Learning & Personalização & User feedback; Statistical; Qualitativa & Ganhos de aprendizagem e engajamento \\
Appiah-Odame (2024) & Avaliação autêntica em matemática & --- & Avaliação/autenticidade & User feedback; Qualitativa & Motivação/eficácia; barreiras: tempo, recursos \\
Mertasari et al. (2023) & Performance assessment e metacognição & --- & Avaliação formativa & User feedback; Statistical & Ganhos metacognitivos: performance > ensaio > múltipla escolha \\
Hasib et al. (2022) & Previsão no secundário com XAI & SVM; K-Means SMOTE; LIME & Predição explicável & Performance; Statistical & SVM 96,89\% acurácia; explicações LIME por classe \\
Kumar et al. (2022) & Seleção de atributos e DM & Learning Analytics; ML & Predição de notas & Performance & DT/JRip/NB/MLP/RF com acurácia razoável \\
Pejic et al. (2021) & PISA: proficiência matemática (3 níveis) & ML & Predição de proficiência & Performance & RNAs e Random Forest previram níveis; métricas Kappa e ROC-AUC \\
Ünal (2020) & DM para previsão de notas & DT; RF; NB & Predição & Experimental & Efetividade demonstrada em dois datasets \\
Salas-Rueda (2021) & Facebook + ML em finanças & Regressão; DT; RN & Apoio ao ensino/aprendizagem & Statistical & Mensagens, vídeos e exercícios correlacionados a ganhos \\
Sokkhey et al. (2020) & Previsão no EM (Camboja) & ML (RF) & Predição & Performance; Statistical & Random Forest maior acurácia e menor MSE \\
Uskov et al. (2019) & Analytics preditiva em STEM & LR; RF; SVM; ANN etc. & Predição & Performance; Statistical & Benchmark de 8 algoritmos; recomendações de uso em sala \\
MacLellan (2017) & Modelos computacionais para tutores & Modelos de aprendiz (DT; TRESTLE) & Tutoria/Autoria de tutores & Statistical & TRESTLE ajustou-se melhor aos dados humanos \\
Depren et al. (2017) & TIMSS 2011 (TR): comparação EDM & LR; DT; BN; RN & Predição/classificação & User feedback; Statistical & Regressão logística superior; confiança do aluno fator saliente \\
\end{longtable}
\end{landscape}

A síntese apresentada na Tabela~\ref{tab:sintese-17-estudos} permite identificar três padrões relevantes para este projeto. Primeiro, a \textbf{predominância de abordagens preditivas}: 11 dos 17 estudos (64,7\%) focam em predição de desempenho ou classificação de proficiência, evidenciando que a comunidade científica tem priorizado a identificação precoce de estudantes em risco. Segundo, a \textbf{diversidade de algoritmos}: embora Random Forest e SVM apareçam com maior frequência, técnicas mais recentes como Deep Learning com Knowledge Graphs \cite{Design2025_004} demonstram potencial para personalização mais sofisticada. Terceiro, a \textbf{consistência metodológica}: todos os estudos empregam métricas de avaliação quantitativas (acurácia, ROC-AUC, testes estatísticos), permitindo comparabilidade entre resultados. Ressalta-se que apenas estudos com pontuação de relevância $\geq 4{,}0$ foram incluídos (média: 4,2; intervalo: 4,0--4,5), garantindo alta aderência aos critérios de qualidade estabelecidos na metodologia.

\section{Análise Temática}

A análise de frequência de termos nos títulos e resumos dos 17 estudos incluídos 
revelou as seguintes tendências principais:

\subsection{Termos Mais Frequentes}

Os cinco termos mais frequentes nos estudos incluídos foram:

\begin{enumerate}
    \item \textbf{Machine Learning} (10 ocorrências --- 58,8\%)
    \item \textbf{Assessment} (9 ocorrências --- 52,9\%)
    \item \textbf{Predictive Analytics} (8 ocorrências --- 47,1\%)
    \item \textbf{Learning Analytics} (7 ocorrências --- 41,2\%)
    \item \textbf{Adaptive Learning} (4 ocorrências --- 23,5\%)
\end{enumerate}

Termos menos frequentes, mas relevantes: \textit{AI/Artificial Intelligence} (2; 11,8\%), \textit{Intelligent Tutoring} (1; 5,9\%), além de abordagens específicas presentes em títulos (e.g., \textit{Reinforcement Learning}, \textit{Knowledge Graph}, \textit{Computer Animation}) que não aparecem como palavras-chave padronizadas.\footnote{As porcentagens referem-se à presença do termo em pelo menos um estudo; termos não listados nos campos \texttt{keywords}, mas citados em títulos ou textos (como \textit{Personalized Learning}), não foram incluídos nesta contagem para manter consistência reprodutível.}

\subsection{Categorias Temáticas Emergentes}

Quatro categorias temáticas principais emergiram da análise qualitativa:

\begin{description}
    \item[Sistemas de Tutoria Inteligente (40\%):] Tutoria adaptativa baseada em 
    modelagem de conhecimento, sistemas de diálogo para suporte ao estudante, 
    scaffolding inteligente, feedback personalizado em tempo real.
    
    \item[Diagnóstico e Avaliação (30\%):] Detecção automatizada de erros e 
    misconceptions, predição de desempenho estudantil, avaliação formativa adaptativa, 
    identificação de lacunas de conhecimento.
    
    \item[Personalização de Conteúdo (20\%):] Sistemas de recomendação de recursos 
    educacionais, geração automática de exercícios, adaptação de dificuldade, 
    trajetórias de aprendizagem individualizadas.
    
    \item[Análise Preditiva (10\%):] Predição de evasão escolar, identificação de 
    estudantes em risco, análise de trajetórias de aprendizagem, modelagem temporal 
    de conhecimento.
\end{description}

\subsection{Distribuição por Abordagem Técnica e Finalidade}

A Tabela~\ref{tab:abordagem-tecnica} sintetiza as abordagens técnicas empregadas 
nos estudos incluídos. Categorias raras (Knowledge Graph, Sentiment Analysis, Computer Animation, Jigsaw Cooperative Learning, ACO, CRF, LIME) foram agrupadas em ``Outras/Específicas''. 

\begin{table}[H]
\centering
\caption{Distribuição de abordagens técnicas}
\label{tab:abordagem-tecnica}
\begin{tabular}{lcc}
\hline
\textbf{Abordagem} & \textbf{N$^{\circ}$ Estudos} & \textbf{\%} \\
\hline
Machine Learning Supervisionado & 13 & 76,5\% \\
Deep Learning (CNN/DL) & 2 & 11,8\% \\
Reinforcement Learning & 1 & 5,9\% \\
Adaptive Learning & 1 & 5,9\% \\
Modelagem de Conhecimento/Tutores (Aprendiz) & 1 & 5,9\% \\
Outras/Específicas\footnotemark & 6 & 35,3\% \\
\hline
\end{tabular}
\end{table}
\footnotetext{ACO, CRF, Knowledge Graph, Sentiment Analysis, Computer Animation, Jigsaw Cooperative Learning, XAI (inclui LIME).}

Os dados da Tabela~\ref{tab:abordagem-tecnica} evidenciam a hegemonia do Machine Learning Supervisionado, presente em 76,5\% dos estudos, o que reflete a maturidade dessas técnicas e a disponibilidade de dados rotulados em contextos educacionais. Deep Learning e Reinforcement Learning aparecem em menor proporção, sugerindo que abordagens mais complexas ainda estão em estágio inicial de adoção na área. A Figura~\ref{fig:techniques-distribution} visualiza essa distribuição, tornando evidente a concentração em técnicas tradicionais de ML.\footnote{A contagem considera presença binária da categoria por estudo; as categorias não são mutuamente exclusivas, de modo que os percentuais podem somar mais de 100\%. A categoria ``Outras/Específicas'' agrega técnicas singulares de baixa frequência.}

\begin{figure}[htbp]
\centering
\includegraphics[width=0.85\textwidth]{images/techniques_distribution.png}
\caption{Distribuição de técnicas de IA nos estudos incluídos.}
\label{fig:techniques-distribution}
\end{figure}

A Tabela~\ref{tab:finalidade-pedagogica} apresenta a distribuição por finalidade 
pedagógica:

\begin{table}[H]
\centering
\caption{Distribuição dos estudos por finalidade principal.}
\label{tab:finalidade-pedagogica}
\begin{tabular}{lcc}
\hline
\textbf{Finalidade Principal} & \textbf{N$^{\circ}$ Estudos} & \textbf{\%} \\
\hline
Predição / Modelagem de Desempenho & 9 & 52,9\% \\
Personalização / Trajetórias Adaptativas & 3 & 17,6\% \\
Ensino / Suporte Instrucional & 2 & 11,8\% \\
Avaliação / Assessment Formativo & 2 & 11,8\% \\
Tutoria Inteligente / Autoria de Tutores & 1 & 5,9\% \\
\hline
\end{tabular}
\end{table}
Os resultados da Tabela~\ref{tab:finalidade-pedagogica} revelam que mais da metade dos estudos (52,9\%) concentra-se em predição e modelagem de desempenho, refletindo o interesse da comunidade científica em antecipar resultados acadêmicos para intervenções preventivas. A personalização de trajetórias adaptativas (17,6\%) e as abordagens de avaliação formativa (11,8\%) aparecem como áreas emergentes com potencial de crescimento.\footnote{Cada estudo foi classificado em uma única finalidade para preservação de proporções; estudos com descrições híbridas receberam atribuição pela finalidade predominante explícita.}

A Figura~\ref{fig:papers-by-year} apresenta a distribuição temporal dos estudos incluídos. Observa-se uma concentração de publicações nos anos mais recentes (2023--2025), indicando crescente interesse da comunidade científica no tema. Esta tendência ascendente sugere que a aplicação de técnicas computacionais na educação matemática é um campo em expansão ativa.

\begin{figure}[H]
\centering
\includegraphics[width=0.85\textwidth]{images/papers_by_year.png}
\caption{Distribuição temporal dos estudos incluídos (2015--2025).}
\label{fig:papers-by-year}
\end{figure}

\section{Resultados de Eficácia Reportados}

Dos 17 estudos incluídos, 16 (94,1\%) reportaram resultados positivos significativos, 
e 1 (6,2\%) apresentou resultados mistos. Nenhum estudo reportou ausência de efeito, 
indicando possível viés de publicação (\textit{publication bias}).

\subsection{Magnitude de Efeito}

A análise preliminar das magnitudes de efeito reportadas indica:

\begin{itemize}
    \item \textbf{Pequeno efeito} (5--10\%): $\sim$37\% dos estudos
    \item \textbf{Médio efeito} (10--20\%): $\sim$38\% dos estudos
    \item \textbf{Grande efeito} ($>20\%$): $\sim$19\% dos estudos
    \item \textbf{Não especificado}: $\sim$6\% dos estudos
\end{itemize}

O possível viés de publicação pode superestimar as estimativas de eficácia; portanto, é necessária análise crítica das limitações metodológicas dos estudos.

A Figura~\ref{fig:relevance-distribution} apresenta a distribuição das pontuações de relevância dos 17 estudos incluídos. A concentração das pontuações próximas ao limiar de 4,0 (média de 4,2; intervalo 4,0--4,5) indica que os estudos selecionados formam um conjunto homogêneo em termos de aderência aos critérios estabelecidos, sem \textit{outliers} que pudessem distorcer a análise qualitativa.

\begin{figure}[H]
\centering
\includegraphics[width=0.85\textwidth]{images/relevance_distribution.png}
\caption{Distribuição das pontuações de relevância dos estudos incluídos.}
\label{fig:relevance-distribution}
\end{figure}

\subsection{Métricas de Avaliação Empregadas}

Os estudos empregaram as seguintes métricas de avaliação:

\begin{description}
    \item[Métricas de Aprendizagem:] Ganhos em testes pré/pós (70\% dos estudos), 
    melhoria em notas acadêmicas (55\%), redução de erros/misconceptions (40\%), 
    tempo para domínio de competências (30\%).
    
    \item[Métricas de Engajamento:] Tempo de uso do sistema (60\%), taxa de conclusão 
    de atividades (45\%), satisfação do usuário (35\%), motivação autorreportada (25\%).
\end{description}

\section{Limitações Identificadas}

A análise crítica dos 17 estudos incluídos revelou quatro categorias principais 
de limitações:

\subsection{Limitações Técnicas}

\begin{enumerate}
    \item \textbf{Dependência de dados rotulados}: Necessidade de grandes volumes 
    de dados anotados (problema de \textit{cold start}).
    \item \textbf{Generalização limitada}: Modelos treinados em contextos específicos 
    não transferem bem para novos contextos.
    \item \textbf{Complexidade computacional}: Alguns modelos requerem recursos 
    computacionais significativos (GPU, memória).
    \item \textbf{Drift temporal}: Modelos degradam com mudanças curriculares ou 
    populacionais.
\end{enumerate}

\subsection{Limitações Pedagógicas}

\begin{enumerate}
    \item \textbf{Foco em conhecimento declarativo}: Pouca atenção a habilidades 
    procedurais e metacognitivas.
    \item \textbf{Simplificação do processo de ensino}: Redução da complexidade 
    pedagógica a variáveis quantificáveis.
    \item \textbf{Desalinhamento curricular}: Sistemas não alinhados a currículos 
    nacionais (ex: BNCC no Brasil).
    \item \textbf{Desconsideração de fatores socioemocionais}: Foco excessivo em 
    desempenho cognitivo.
\end{enumerate}

\subsection{Limitações Metodológicas}

\begin{enumerate}
    \item \textbf{Viés de publicação}: Predominância de resultados positivos (95\% 
    dos estudos).
    \item \textbf{Falta de grupo controle}: Muitos estudos sem comparação rigorosa 
    (45\%).
    \item \textbf{Tamanhos de amostra pequenos}: Limitação de poder estatístico 
    (35\% com $n<30$).
    \item \textbf{Ausência de dados abertos}: Dificuldade de replicação (apenas 20\% 
    compartilham dados).
    \item \textbf{Heterogeneidade metodológica}: Dificuldade de síntese quantitativa 
    (meta-análise).
\end{enumerate}

\subsection{Limitações Éticas}

\begin{enumerate}
    \item \textbf{Privacidade de dados}: Poucos estudos discutem proteção de dados 
    estudantis (LGPD, GDPR).
    \item \textbf{Viés algorítmico}: Escassa análise de equidade e justiça dos 
    sistemas (10\%).
    \item \textbf{Consentimento}: Procedimentos de consentimento informado raramente 
    detalhados (30\%).
    \item \textbf{Transparência}: Falta de explicabilidade dos modelos (15\%).
\end{enumerate}

\section{Mapeamento de Lacunas e Direcionamento para Fase 2}

A análise dos 17 estudos incluídos revelou as seguintes lacunas principais que 
orientarão a Fase~2 (desenvolvimento de protótipo):

\subsection{Lacunas Técnicas}

\begin{itemize}
    \item \textbf{Escassez de validação ecológica}: Apenas 35\% dos estudos reportam 
    validação em contextos educacionais reais (escolas, universidades).
    \item \textbf{Limitações de interpretabilidade}: Poucos estudos (15\%) abordam 
    explicabilidade de modelos de IA.
    \item \textbf{Ausência de estudos longitudinais}: A maioria dos estudos tem 
    duração limitada ($<1$ semestre).
\end{itemize}

\subsection{Lacunas Pedagógicas}

\begin{itemize}
    \item \textbf{Desalinhamento curricular}: Necessidade de sistemas alinhados à 
    BNCC (Base Nacional Comum Curricular) brasileira.
    \item \textbf{Foco limitado em metacognição}: Poucos sistemas abordam habilidades 
    metacognitivas e autorregulação.
    \item \textbf{Integração com práticas docentes}: Falta de suporte para apropriação 
    pedagógica pelos professores.
\end{itemize}

\subsection{Direcionamento para Fase 2}

Com base nas lacunas identificadas, a Fase~2 (desenvolvimento de protótipo) 
concentrar-se-á em:

\begin{enumerate}
    \item \textbf{Sistema explicável}: Desenvolvimento de sistema de IA com 
    explicabilidade integrada (XAI).
    \item \textbf{Alinhamento curricular}: Adequação à BNCC, especialmente 
    competências matemáticas do Ensino Fundamental~II e Ensino Médio.
    \item \textbf{Validação ecológica}: Planejamento de experimentos em contexto 
    escolar real com grupo controle.
    \item \textbf{Foco em metacognição}: Incorporação de estratégias de autorregulação 
    e feedback metacognitivo.
    \item \textbf{Dados abertos e reprodutibilidade}: Compartilhamento de código, 
    dados e modelos treinados.
\end{enumerate}

\section{Próximos Passos}

A conclusão da Fase~1 (Projeto de TCC --- PTC) com esta revisão sistemática estabelece 
fundação sólida para as fases subsequentes:

\subsection{Fase 2 --- Desenvolvimento de Protótipo (fevereiro--julho/2026)}

\begin{itemize}
    \item Levantamento de requisitos baseado nas lacunas identificadas
    \item Desenvolvimento de protótipo de sistema explicável
    \item Alinhamento com competências BNCC
    \item Implementação de componentes de autorregulação
    \item Testes unitários e de integração
\end{itemize}

\subsection{Fase 3 --- Validação Experimental (julho--novembro/2026)}

\begin{itemize}
    \item Desenho experimental com grupo controle
    \item Execução de experimentos em contexto escolar real
    \item Coleta de dados quantitativos e qualitativos
    \item Análise estatística de resultados
    \item Discussão de implicações para prática e política educacional
\end{itemize}

\subsection{Síntese Final do TCC (outubro--novembro/2026)}

\begin{itemize}
    \item Integração de resultados das três fases
    \item Discussão de contribuições teóricas e práticas
    \item Recomendações para pesquisas futuras
    \item Submissão e defesa do TCC
\end{itemize}